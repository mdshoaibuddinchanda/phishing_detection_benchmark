% IEEE Conference Paper Template
% Replace with actual IEEE template from conference

\documentclass[conference]{IEEEtran}

\usepackage{cite}
\usepackage{amsmath,amssymb,amsfonts}
\usepackage{algorithmic}
\usepackage{graphicx}
\usepackage{textcomp}
\usepackage{xcolor}
\usepackage{booktabs}
\usepackage{hyperref}

\begin{document}

\title{Accuracy–Energy Trade-offs of Small Language Models for Real-Time Phishing Detection}

\author{
\IEEEauthorblockN{Your Name}
\IEEEauthorblockA{\textit{Department} \\
\textit{University}\\
City, Country \\
email@university.edu}
}

\maketitle

\begin{abstract}
Phishing detection using deep learning models has achieved high accuracy but at significant computational cost. We empirically evaluate the accuracy–energy trade-offs of Small Language Models (SLMs) compared to large transformer models for phishing email classification. Our benchmark compares RoBERTa-Large, DistilBERT, and Phi-3-mini across performance metrics (accuracy, precision, recall, F1-score) and efficiency metrics (energy consumption, CO₂ emissions, inference latency, model size). Results demonstrate that SLMs achieve near-equivalent performance (F1-score within 2–3\%) while reducing energy consumption by up to XX\% and inference latency by XX\%. These findings support the feasibility of deploying energy-efficient phishing detection systems without sacrificing accuracy, advancing sustainable AI practices in cybersecurity.
\end{abstract}

\begin{IEEEkeywords}
phishing detection, small language models, energy efficiency, green AI, transformer models, cybersecurity
\end{IEEEkeywords}

\section{Introduction}
% Background on phishing
% Deep learning effectiveness
% Energy efficiency gap
% Research contribution

\section{Related Work}
\subsection{Phishing Detection with NLP}
\subsection{Transformer-Based Classifiers}
\subsection{Green AI and Energy-Efficient ML}

\section{Methodology}
\subsection{Dataset}
\subsection{Model Selection}
\subsection{Training Configuration}
\subsection{Energy Measurement}
\subsection{Evaluation Metrics}

\section{Experimental Results}

\subsection{Performance Metrics}
% Table 1: Complete metrics comparison
\begin{table}[htbp]
\caption{Performance and Efficiency Metrics}
\begin{center}
\begin{tabular}{lcccccc}
\toprule
\textbf{Model} & \textbf{Acc} & \textbf{F1} & \textbf{Size} & \textbf{Latency} & \textbf{Energy} & \textbf{CO₂} \\
 & & & (MB) & (ms) & (kWh) & (g) \\
\midrule
RoBERTa & XX & XX & XX & XX & XX & XX \\
DistilBERT & XX & XX & XX & XX & XX & XX \\
Phi-3-mini & XX & XX & XX & XX & XX & XX \\
\bottomrule
\end{tabular}
\label{tab:results}
\end{center}
\end{table}

\subsection{Accuracy-Energy Trade-offs}
% Figure 1: Pareto frontier
\begin{figure}[htbp]
\centerline{\includegraphics[width=\columnwidth]{figures/pareto_frontier.png}}
\caption{Accuracy–Energy Pareto Frontier}
\label{fig:pareto}
\end{figure}

\section{Discussion}
\subsection{Performance Comparison}
\subsection{Energy Efficiency Analysis}
\subsection{Practical Implications}

\section{Limitations}
% Single task focus
% Dataset scope
% Hardware specificity

\section{Conclusion}
% Restate findings
% Sustainability argument
% Future directions

\bibliographystyle{IEEEtran}
\bibliography{references}

\end{document}
